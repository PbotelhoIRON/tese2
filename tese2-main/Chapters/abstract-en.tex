%!TEX root = ../template.tex
%%%%%%%%%%%%%%%%%%%%%%%%%%%%%%%%%%%%%%%%%%%%%%%%%%%%%%%%%%%%%%%%%%%%
%% abstrac-en.tex
%% NOVA thesis document file
%%
%% Abstract in English([^%]*)
%%%%%%%%%%%%%%%%%%%%%%%%%%%%%%%%%%%%%%%%%%%%%%%%%%%%%%%%%%%%%%%%%%%%

\typeout{NT FILE abstrac-en.tex}%

Regardless of the language in which the dissertation is written, a summary is required in the same language as the main text and another summary in another language. It is assumed that the two languages in question are Portuguese and English.

The abstracts should appear first in the language of the main text and then in the other language. For example, if the dissertation is written in Portuguese the abstract in Portuguese will appear first, then the abstract in English, followed by the main text in Portuguese. If the dissertation is written in English, the abstract in English will appear first, then the abstract in Portuguese, followed by the main text in English. 

In the \LaTeX\ version, the NOVAthesis template will automatically order the two abstracts taking into account the language of the main text. You may change this behaviour by adding
\begin{verbatim}
    \abstractorder(<MAIN_LANG>):={<LANG_1>,...,<LANG_N>}
\end{verbatim}
\noindent to the customization area in the document preamble, e.g.,
\begin{verbatim}
    \abstractorder(de):={de,en,it}
\end{verbatim}

The abstracts should not exceed one page and, in a generic way, should answer the following questions (it is essential to adapt to the usual practices of your scientific area):

\begin{enumerate}
  \item What is the problem?
  \item Why is this problem interesting/challenging?
  \item What is the proposed approach/solution?
  \item What results (implications/consequences) from the solution?
\end{enumerate}

% Palavras-chave do resumo em Inglês
\begin{keywords}
Keyword 1, Keyword 2, Keyword 3, Keyword 4, Keyword 5, Keyword 6, Keyword 7, Keyword 8, Keyword 9
\end{keywords} 
