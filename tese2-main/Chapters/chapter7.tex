%!TEX root = ../template.tex
%%%%%%%%%%%%%%%%%%%%%%%%%%%%%%%%%%%%%%%%%%%%%%%%%%%%%%%%%%%%%%%%%%%%
%% chapter4.tex
%% NOVA thesis document file
%%
%% Chapter with lots of dummy text
%%%%%%%%%%%%%%%%%%%%%%%%%%%%%%%%%%%%%%%%%%%%%%%%%%%%%%%%%%%%%%%%%%%%

\typeout{NT FILE chapter4.tex}%

\chapter{Design and Application of Tooling for CMM: Fixation System Modeling}
\label{cha:dig}

In this chapter, the work previously developed in the theses of António Guerreiro and Rendas will be reviewed, as they provided the foundation for the fixation system design. 
Following this, the prototype developed will be presented, with a focus on the main differences between this new design and the previous ones. 
The analyses and calculations made during the development of the prototype will then be discussed, providing insights into the technical decisions and challenges encountered. 
The production process of the prototypes will be described, starting with 3D printing using PLA, followed by machining on the CNC machine at the faculty. 
Finally, the application of the prototype in the TAP CMM system will be demonstrated, showcasing its practical implementation in the measurement environment.

\section{Past Work}
\label{sec:pastfix}

In the chapter "Fixture Prototyping," Pedro Rendas describes the development of a custom fixture designed to support and accurately position compressor blades during coordinate measuring machine (CMM) inspections.
The main goals of the fixture were to ensure repeatable and stable positioning of the blade, provide unobstructed access to critical measurement areas, and allow alignment with the blade’s functional geometric references, such as support planes and the central axis.

The development process began with a geometric analysis of the blade to determine the most appropriate support points. 
This led to the design and fabrication of a prototype featuring a fixed base where the blade rests on two defined planes and aligns laterally with a reference surface. 
The setup was complemented by mobile elements, including clamping screws and grips, to securely lock the blade in place during measurements.

Validation tests on the CMM confirmed that the fixture provided both stability and repeatability, key requirements for reliable dimensional inspection. 

The author emphasizes the importance of incorporating geometric tolerances and stability criteria into the design, ensuring the fixture could accommodate the blade's geometry while remaining compatible with the CMM's kinematics and measurement objectives.


\section{Prototipe}
\label{sec:proto}