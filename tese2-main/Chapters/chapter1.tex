%!TEX root = ../template.tex
%%%%%%%%%%%%%%%%%%%%%%%%%%%%%%%%%%%%%%%%%%%%%%%%%%%%%%%%%%%%%%%%%%%
%% chapter1.tex
%% NOVA thesis document file
%%
%% Chapter with introduction
%%%%%%%%%%%%%%%%%%%%%%%%%%%%%%%%%%%%%%%%%%%%%%%%%%%%%%%%%%%%%%%%%%%

\typeout{NT FILE chapter1.tex}%

\chapter{Introduction}
\label{cha:introduction}


\section{Motivation}
\label{sec:motivation}

In 2016, CFMI launched the LEAP-1A engine, ushering in a new era of efficiency and performance for commercial aviation. This engine builds on the solid foundation laid by the CFM56, which has been one of the most trusted and widely used engines in the industry.

Over the course of its life, an engine undergoes numerous upgrades and refinements aimed at improving its performance and fuel efficiency. These improvements often focus on precise measurements and dimensional control, as well as maintenance standards that ensure the engine continues to run smoothly and reliably. By implementing these updates, the engine can perform at its best throughout its service life, maximizing efficiency and reducing operational costs. Additionally, these advancements are aligned with the growing need for more sustainable aviation technologies, helping to meet the industry's environmental goals.

Each time an engine manufacturer introduces an optimization, it is implemented through a \gls{SB}, a document that communicates details of modifications that can be made to the aircraft.

In recent years, two students, Edgar Farinhas and Pedro Rendas, explored a method for measuring the rotor blade dimensions of the \gls{HPC} using a \gls{CMM} machine, focusing specifically on the blades of the CFM56 \gls{HPC}. \cite{Farinha2021}

Since it entered service, \gls{TAP} has integrated the \gls{LEAP}-1A into approximately half of its fleet, replacing the CFM56. As the performance of the \gls{HPC} directly influences engine efficiency and overall operational effectiveness, establishing robust monitoring and control measures has become a critical priority, alongside investigating the factors that affect its performance. 


\section{Objectives}
\label{sec:objectives}

Various factors influence engine performance, and the high-pressure compressor (HPC) plays a key role in this. The efficiency of the HPC is largely determined by the design of its rotor blades, which operate under intense aerodynamic and thermal conditions. This directly impacts the engine’s performance and fuel consumption. Despite their importance, TAP has no standardized dimensional inspection process for these components.

Using the available equipment at the TAP Engine Shop, including the 3D scanner and Coordinate Measuring Machine (CMM), this study aims to develop a practical and efficient method for measuring the chord length of HPC blades, as this parameter is crucial in assessing blade wear and its correlation with engine performance in test cell conditions. Additionally, the study seeks to create a tool to measure the total clearance of the entire stage after assembly, optimizing the assembly process and making it more efficient.

An important objective is to control the platform gap of the blades during their preparation for assembly, developing a tool capable of accurately simulating the blade fit. This will enable the immediate determination of the required number of wide and narrow platform blades, making the process faster and more efficient—an important consideration for TAP’s operational needs.

By establishing this methodology, it will be possible to correlate these geometric characteristics with engine performance as tested in the test cell. Understanding these relationships will help optimize maintenance procedures and improve overall efficiency. This thesis represents an important step toward implementing a more advanced dimensional inspection process at TAP, contributing to the company’s ongoing efforts to enhance engine performance and reliability.



%\prependtographicspath{{Chapters/Figures/Covers/}}

% epigraph configuration
%\epigraphfontsize{\small\itshape}
%\setlength\epigraphwidth{12.5cm}
%\setlength\epigraphrule{0pt}

%\section{If You Use this Template…} 
%\label{sub:if_you_use_this_template}


%\subsection{Your Time is Precious}
%\label{sub:time_is_money}

%\subsection{Appeal}
%\label{sub:appeal}

%\section{The \emph{NOVAthesis} template}
%\label{sec:a_bit_of_history}

%\section{Getting Started}
%\label{sec:getting_started}

%\subsection{Using Overleaf}
%\label{sub:using_overleaf}

%\subsection{Using a Local \LaTeX\ Installation}
%\label{sub:using_local_latex}

%\section{Getting Help}
%\label{sec:getting_help}

%\section{Reporting Problems}
%\label{sec:reporting_problems}

%\section{Donors}

%\section{Disclaimer}
%\label{sec:disclaimer}
